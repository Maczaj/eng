\documentclass[11pt,a4paper]{article}
 
\usepackage{polski}
\usepackage[utf8]{inputenc} %utf-8 żeby nie było problemów z przenoszeniem między windowsem a linuxem
\usepackage{graphicx}
\usepackage{listings}
\usepackage{indentfirst}

%lokalne makro na komendę
\newcommand{\HRule}{\rule{\linewidth}{0.5mm}}

\setlength\parindent{24pt}
\oddsidemargin=20pt
\textwidth=420pt



 
\begin{document}

%strona tytułowa
\begin{titlepage}
\begin{center}
%nagłówek uczelni
\textsc{\LARGE Politechnika Warszawska}\\[0.5cm]
\textsc{\Large Wydział Elektroniki i Technik Informacyjnych}\\
\textsc{\Large Instytut Automatyki i Informatyki Stosowanej}\\[1.5cm]

\textsc{\Large Sprawozdanie z pracowni dyplomowej inżynierskiej II} \\[0.5cm]
%tytuł między kreskami
\HRule \\[0.4cm]
{ \huge \bfseries Generator raportów z baz danych w technologii XSL-FO  bazujący na wzorcach \\[0.4cm]}
\HRule \\[1.5cm]

%autor
\begin{minipage} {0.4\textwidth}
\begin{flushleft} \large
\emph{Autor:}\\
 Maciej Kucharski 
\end{flushleft}
\end{minipage}
%promotor
\begin{minipage} {0.4\textwidth}
\begin{flushright} \large

\emph{Opiekun:}\\
doc. dr inż. Tomasz Traczyk

\end{flushright}
\end{minipage}	
\vfill
\large Semestr 14Z

	\end{center}
\end{titlepage}
 

\newpage

\section{Streszczenie} \label{sec:wst}
Obecnie bazy danych wykorzystywane są powszechnie. Niemal każde przedsiębiorstwo dysponujące przynajmniej podstawową infrastrukturą informatyczną wykorzystuje w swoich działaniach bazę danych. Dane zgromadzone w takiej bazie danych często muszą być przedstawiane, np. zarządowi przedsiębiorstwa, bądź udostępniane publicznie w przypadku instytucji publicznych. Zwykła, surowa tabela będąca wynikiem działania zapytania \emph{SQL} nie jest wystarczająca dla takich zastosowań ze względu na jej ubogi wygląd oraz kiepską czytelność.  Samo pobieranie raportowanych danych z bazy również jest problematyczne ze względu na konieczność znajomości języka \emph{SQL}. 

Funkcjonalny z punktu widzenia biznesu system raportowy powinien spełniać pewne wymagania:
\begin{itemize}
	\item umożliwienie stosowania własnych wzorców wyglądu raportu (tzw. \emph{layout}),
	\item minimalizacja wymaganej do obsługi ilości wiedzy informatycznej ,
	\item umożliwienie generowania raportów w formacie \emph{PDF} w celu uniknięcia problemów wynikających z innego wyświetlania na różnych platformach systemowych.
\end{itemize}

Na rynku dostępna jest cała gama narzędzi spełniających powyższe wymagania w mniejszym lub większym stopniu. Najlepszym przykładem jest \emph{Oracle BI Publisher}, który spełnia wszystkie wyżej postawione wymagania, jednak konieczność corocznego ponoszenia niemałych kosztów wynikających z opłat licencyjnych sprawia, że jest on w zasięgu niewielu organizacji.

Alternatywą jest darmowy \emph{Oracle Application Express}, zwany również \emph{ApEx'em}, instalujący się wraz z bazą danych \emph{Oracle}. Standardowo generowane przez \emph{ApEx} raporty nie są zbyt zaawansowane graficznie. Celem niniejszej pracy jest opracowanie rozwiązań zwiększających jego możliwości.

\emph{ApEx} umożliwia skorzystanie z zewnętrznego serwera wydruków do tworzenia raportów w formacie \emph{PDF}. W tym celu potrzebny jest raport w postaci \emph{XSL-FO}. Postać tę można uzyskać poprzez zastosowanie transformacji \emph{XSLT} do danych w formacie \emph{XML}. Jest to jednak kłopotliwe w przypadku ręcznego tworzenia arkuszy transformacji \emph{XSLT}, gdyż zmiany we wzorcu wyglądu raportu pociągają za sobą konieczność wykorzystania wiedzy eksperckiej w celu modyfikacji arkusza \emph{XSLT}.

\bigskip
W ramach pracy przygotowano:
\begin{itemize}
	\item narzędzie generujące arkusz przekształceń \emph{XSLT} na podstawie danych wejściowych i wzorca wyglądu raportu,
	\item przykładową konfigurację \emph{ApEx},
	\item przykładowe wzorce wyglądu raportu w celu demonstracji.
\end{itemize}

\end{document}

